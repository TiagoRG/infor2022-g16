\documentclass{report}
\usepackage[T1]{fontenc} % Fontes T1
\usepackage[utf8]{inputenc} % Input UTF8
\usepackage[backend=biber, style=ieee]{biblatex} % para usar bibliografia
\usepackage{csquotes}
\usepackage[portuguese]{babel} %Usar língua portuguesa
\usepackage{blindtext} % Gerar texto automaticamente
\usepackage{hyperref} % para autoref
\usepackage{graphicx}
\usepackage{indentfirst}
\usepackage{acronym}

\bibstyle{plain}
\bibliography{bibliografia}


\begin{document}
%%
% Definições
%
\def\titulo{CIBERSEGURANÇA}
\def\data{DATA}
\def\autores{Rúben Gomes, Tiago Garcia}
\def\autoresacronimos{RG, TG}
\def\autorescontactos{113435 rlcg@ua.pt, 114184 tiago.rgarcia@ua.pt}
\def\versao{VERSÃO 1}
\def\departamento{Dept. de Eletrónica, Telecomunicações e Informática}
\def\empresa{Universidade de Aveiro}
\def\logotipo{ua.pdf}

%
%%%%%% CAPA %%%%%%
%
\begin{titlepage}

\begin{center}
%
\vspace*{50mm}
%
{\Huge \titulo}\\ 
%
\vspace{10mm}
%
{\Large \empresa}\\
%
\vspace{10mm}
%
{\LARGE \autores}\\ 
%
\vspace{30mm}
%
\begin{figure}[h]
    \center
    \includegraphics{ua}
\end{figure}
%
\vspace{30mm}
\end{center}
%
\begin{flushright}
\versao
\end{flushright}
\end{titlepage}

%%  Página de Título %%
\title{%
{\Huge\textbf{\titulo}}\\
{\Large \departamento\\ \empresa}
}
%
\author{%
    \autores \\
    \autorescontactos
}
%
\date{\today}
%
\maketitle

\pagenumbering{roman}

%%%%%% RESUMO %%%%%%
\begin{abstract}
Resumo de 200~-~300 palavras.
\end{abstract}

%%%%%% Agradecimentos %%%%%%
Segundo glisc deveria aparecer após conclusão
\renewcommand{\abstractname}{Agradecimentos}
\begin{abstract}
Eventuais agradecimentos.
Comentar bloco caso não existam agradecimentos a fazer.
\end{abstract}


\tableofcontents
% \listoftables     % descomentar se necessário
% \listoffigures    % descomentar se necessário


%%%%%%%%%%%%%%%%%%%%%%%%%%%%%%%
\clearpage
\pagenumbering{arabic}

%%%%%%%%%%%%%%%%%%%%%%%%%%%%%%%%
\chapter*{Introdução}
\label{chap.introducao}

Com o passar dos anos, as tecnologias que temos ao nosso dispor tem evoluído de uma forma rápida e sem fim, como, por exemplo, o \textit{hardware} de um computador.\ Mas, com constantes evoluções, também vem uma necessidade de responsabilidade, pois com quanto mais recursos existirem, maior será o impacto de danos a indivíduos ou entidades.\\

Com este relatório, iremos falar num tópico bastante sensível na atualidade, a Cibersegurança, bem como as diversas vulnerabilidades associadas de formas de nos protegermos contra as mesmas.

\chapter{Cibersegurança no geral}
\label{chap.cibersegurancanogeral}
\section{Segurança}

\chapter{Legislação}
\label{chap.legislacao}
\section{Legislação}
\section{Proteção de dados}

\chapter{Programação aplicada à cibersegurança}
\label{chap.programacao}

\chapter{Vulnerabilidades}
\label{chap.vulnerabilidade}
\section{Análise de vulnerabilidades}
\section{Análise de evidências}

\chapter{Conclusões}
\label{chap.conclusao}
Apresenta conclusões.

\chapter*{Contribuições dos autores}
Resumir aqui o que cada autor fez no trabalho.
Usar abreviaturas para identificar os autores,
por exemplo AS para António Silva.

\vspace{10pt}
\textbf{Indicar a percentagem de contribuição de cada autor.}\\

\autoresacronimos : xx\%, yy\%\\

%%%%%%%%%%%%%%%%%%%%%%%%%%%%%%%%%

\chapter*{Acrónimos}
\begin{acronym}
\acro{ua}[UA]{Universidade de Aveiro}
\acro{leci}[LECI]{Licenciatura em Engenharia de Computadores e Informática}
\acro{glisc}[GLISC]{Grey Literature International Steering Committee}
\end{acronym}


%%%%%%%%%%%%%%%%%%%%%%%%%%%%%%%%%
\printbibliography

\end{document}
