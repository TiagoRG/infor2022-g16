\documentclass{report}
\usepackage[T1]{fontenc} % Fontes T1
\usepackage[utf8]{inputenc} % Input UTF8
\usepackage[backend=biber, style=ieee]{biblatex} % para usar bibliografia
\usepackage{csquotes}
\usepackage[portuguese]{babel} %Usar língua portuguesa
\usepackage{blindtext} % Gerar texto automaticamente
\usepackage{hyperref} % para autoref
\usepackage{graphicx}
\usepackage{indentfirst}
\usepackage[printonlyused]{acronym}

\bibliographystyle{plain}
\bibliography{bibliografia}


\begin{document}
%%
% Definições
%
\def\titulo{CIBERSEGURANÇA}
\def\data{DATA}
\def\autores{Rúben Gomes, Tiago Garcia}
\def\autorescontactos{113435 rlcg@ua.pt, 114184 tiago.rgarcia@ua.pt}
\def\versao{VERSÃO 1}
\def\departamento{Dept. de Eletrónica, Telecomunicações e Informática}
\def\empresa{Universidade de Aveiro}
\def\logotipo{ua.pdf}

%
%%%%%% CAPA %%%%%%
%
\begin{titlepage}

\begin{center}
%
\vspace*{50mm}
%
{\Huge \titulo}\\ 
%
\vspace{10mm}
%
{\Large \empresa}\\
%
\vspace{10mm}
%
{\LARGE \autores}\\ 
%
\vspace{30mm}
%
\begin{figure}[h]
    \center
    \includegraphics{ua}
\end{figure}
%
\vspace{30mm}
\end{center}
%
\begin{flushright}
\versao
\end{flushright}
\end{titlepage}

%%  Página de Título %%
\title{%
{\Huge\textbf{\titulo}}\\
{\Large \departamento\\ \empresa}
}
%
\author{%
    \autores \\
    \autorescontactos
}
%
\date{\today}
%
\maketitle

\pagenumbering{roman}

%%%%%% RESUMO %%%%%%
\begin{abstract}
Resumo de 200~-~300 palavras.
\end{abstract}

%%%%%% Agradecimentos %%%%%%
%Segundo glisc deveria aparecer após conclusão
%\renewcommand{\abstractname}{Agradecimentos}
%\begin{abstract}
%Eventuais agradecimentos.
%Comentar bloco caso não existam agradecimentos a fazer.
%\end{abstract}


\tableofcontents
% \listoftables     % descomentar se necessário
% \listoffigures    % descomentar se necessário


%%%%%%%%%%%%%%%%%%%%%%%%%%%%%%%
\clearpage
\pagenumbering{arabic}

%%%%%%%%%%%%%%%%%%%%%%%%%%%%%%%%
\chapter*{Introdução}
\label{chap.introducao}

Com o passar dos anos, as tecnologias que temos ao nosso dispor tem evoluído de uma forma rápida e sem fim, como, por exemplo, o \textit{hardware} de um computador.\ Mas, com constantes evoluções, também vem uma necessidade de responsabilidade, pois com quanto mais recursos existirem, maior será o impacto de danos a indivíduos ou entidades.\\

Com este relatório, irá ser abordado um tópico bastante sensível na atualidade, a Cibersegurança, bem como as diversas vulnerabilidades associadas de formas de nos protegermos contra as mesmas.

\chapter{Cibersegurança no geral}
\label{chap.cibersegurancanogeral}
\section{Segurança}

\chapter{Legislação}
\label{chap.legislacao}
\section{Legislação}
\section{Proteção de dados}

\chapter{Programação aplicada à cibersegurança}
\label{chap.programacao}

\chapter{Vulnerabilidades}
\label{chap.vulnerabilidade}
\section{Análise de vulnerabilidades}
\label{sec:analise-de-vulnerabilidades}
A análise de vulnerabilidades no papel da Cibersegurança é muito importante. \ Sem ela, não existiria nenhuma melhoria no ramo de combate às ameaças cibernéticas, pois sem desenvolvimento, não haveria métodos de nos proteger dos diversos tipos de ataque que se conhece. \par
Este tópico é dividido em vários ramos, como a \textbf{\nameref{subsec:digital-forensics}} e a \textbf{\nameref{subsec:incident-response}}

\subsection{Digital Forensics}
\label{subsec:digital-forensics}
A Digital Forensics é, como o nome indica, a forense digital, e é a obtenção e análise de dados de uma forma pura, sem quais quer tipos de distorção e sem tendências para qualquer lado, de modo a reconstruír o que se passou no passado com o sistema. \par
Tem como objetivo examinar dados de sistemas, ativdidade de utilizadores do sistema, programas em execução, entre outras métricas que possam ajudar a determinar se está a decorrer um ataque e quem está por detrás do ataque. \par
Deve-se ter sempre em conta a preservação dos dados. \ Se o caso a ser estudado for levado a tribunal e se descobrir que os dados foram, de qualquer forma, manipulados, é o suficiente para a prova de esses mesmos dados ser completamente anulada, e até mesmo levar o caso contra quem apresentou essa prova.

\subsection{Incident Response}
\label{subsec:incident-response}


\section{Análise de evidências}
\label{sec:analise-de-evidencias}


\chapter{Conclusões}
\label{chap.conclusao}
Apresenta conclusões.

\chapter*{Contribuições dos autores}
Resumir aqui o que cada autor fez no trabalho.
Usar abreviaturas para identificar os autores,
por exemplo AS para António Silva.

\vspace{10pt}
\textbf{Indicar a percentagem de contribuição de cada autor.}\\

\ac{rg}, \ac{tg} : xx\%, yy\%\\

%%%%%%%%%%%%%%%%%%%%%%%%%%%%%%%%%

\chapter*{Acrónimos}
\begin{acronym}
    \acro{leci}[LECI]{Licenciatura em Engenharia de Computadores e Informática}
    \acro{rg}[RG]{Rúben Gomes}
    \acro{rgpd}[RGPD]{Regulamento Geral sobre a Proteção de Dados}
    \acro{tg}[TG]{Tiago Garcia}
    \acro{ua}[UA]{Universidade de Aveiro}
\end{acronym}


%%%%%%%%%%%%%%%%%%%%%%%%%%%%%%%%%
\printbibliography

\end{document}
